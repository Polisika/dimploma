При генерации контента нейронной сетью полученный результат будет отличаться от продукта человеческой деятельности.

В случае генерации музыки моделью MuseGAN это возможно определить визуально, а также непосредственно послушав композицию. Для визуального определения потребуется отобразить ноты из MIDI-файла: нейронная сеть часто может в один момент заполнить композицию большим количеством нот. А при прослушивании непосредственно композиции из подобного файла можно услышать некоторое нарушение согласованности дорожек между собой.

Контроль результатов работы нейронной сети (хотя бы при применении обычного правила) мог бы повысить качество результата. Это и есть идея модели DALL-E, которая рассмотрена подробнее в данной главе.

\section{ОПИСАНИЕ ЭВРИСТИКИ ОТБОРА}
    В качестве эвристики отбора было выбрано использование модели для классификации жанра композиции. У модели есть некоторая вероятность того, что данная точка относится к тому или иному классу.
    
    Основной идеей данного подхода является следующая гипотеза:
    $$ \exists x_{best} \in X_{model}: \max (p) \to 1, $$
    где $x_{best}$ -- лучшая композиция из сгенерированной выборки моделью $X_{model}$, $p \in \mathbb{R}^{N}$ -- вектор вероятностей принадлежности к тому или иному жанру, при этом $\sum\limits_{i=1}\limits^{N} p_i = 1$, $p = genre(x_{best})$.
    
    Т. е. в данном случае предполагается, что мусорные композиции не имеют отличительного жанра. В таком случае должны быть получены примерно похожие вектора вероятностей $p$ от таких композиций.

\section{ИССЛЕДОВАНИЯ}    
    В рамках исследований были взяты MIDI-файлы, получающиеся в результате работы каждой из моделей и поданы на вход предобученным многоклассовым классификаторам, которые могут определять жанр. В итоге получаем некоторый вектор вероятностей принадлежности полученной композиции к определённому жанру. В данном случае необходимо абстрагироваться от того, какой жанр определяется, т.к. нам необходима лишь некоторая вариативность результата.
    
    Были взяты два классификатора -- один систематизирует признаки, которые получаются в результате анализа MP3, другой -- анализирует MIDI-файл.
    
    Результаты классификации представлены на сайте (описан в главе \ref{chapter31}). Их рассмотрим
    более подробно в \ref{chapter4} главе
    
    Стоит отметить, что обучение обоих классификаторов производилось на качественных размеченных наборах данных. В итоге классификаторы могут с наибольшей уверенностью оценить те элементы, которые удовлетворяют изначальным свойствам набора данных. Перечислим примеры таких свойств. 
    \begin{enumerate}
        \item Строгая фиксацию входного MIDI-файла.
        \item Профессиональное качество записи музыкальной композиции.
        \item Присутствие нескольких инструментов в композиции.
        \item Музыкальный размер композиции 4/4. 
        \item Живая игра на инструментах (velocity -- сила игры различных нот непостоянна).
    \end{enumerate}
    
    Таких свойств много, однако рассмотрены только названные.
    
    В данном случае MuseGAN позволяет гарантировать наличие нескольких инструментов в композиции. А Theme Transformer -- имитировать живую игру на инструменте.
    
    Для MuseGAN и Theme Transformer результаты похожи. В нашем случае модели, скорее всего, в результате работы создавали композиции, максимально похожие на обучающий набор данных и, таким образом, всегда получался некоторый средний жанр.
    
    В итоге, когда модели с точки зрения классификаторов генерируют одно и то же -- стоит использовать не исходную гипотезу о близости вероятности к единице при определении жанра. Возможно пользоваться вектором вероятностей для определения композиции, которая могла бы быть отобрана среди остальных. 

\section{ВАРИАНТЫ РЕАЛИЗАЦИИ}
    Для реализации описанной идеи можно использовать следующие инструменты.
    \begin{enumerate}
        \item Модель обнаружения аномалий, чтобы получать те композиции, которые сильно отличаются от остальных. 
        \item Кластеризация и дальнейшая интерпретация её результатов при предварительном размещении получившейся композиции.
    \end{enumerate}
    
    В данной работе названные варианты не реализованы и являются лишь предположением, которые можно проверить.
