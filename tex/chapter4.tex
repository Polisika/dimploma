В рамках исследований рассматриваются только модели с использованием библиотеки для глубокого обучения PyTorch. 

Для изучения моделей необходимо знать их параметры. В открытом доступе были представлены параметры обученных моделей, однако было решено попробовать воспроизвести результаты самостоятельно. Результаты были успешно воспроизведены.  

Обучение моделей происходит на кластере Математического центра Академгородка со следующими характеристиками.
\begin{enumerate}
    \item Процессор Intel(R) Xeon(R) Gold 6148 CPU @ 2.40GHz.
    \item 754 ГБ оперативной памяти.
    \item 8 видеокарт Tesla V100-SXM2-32GB.
    \item Дистрибутив Red Gat Enterprise Linux Server release 7.6 (Maipo).
\end{enumerate}

Рассмотрим процесс обучения двух моделей и попробуем соединить их, чтобы в итоге получалась генерация композиции с нуля.

В качестве исследований рассмотрено, что учит модель, что меняется с ростом количества эпох и что получается в результате определения жанра.

Анализ проводился на основе композиций, представленных в главе в соответствующих разделах на сайте (описан в главе \ref{chapter31}).

\section{MUSEGAN}
Т. к. исходная модель реализована с помощью библиотеки для глубокого обучения TensorFlow, то необходимо было искать альтернативную реализацию на PyTorch \cite{musegan}. Для данной модели была найдена её реализация через PyTorch от того же автора \cite{muse-torch}. 

\subsection{ПРОЦЕСС ОБУЧЕНИЯ}
Данная модель обучалась с использованием одной видеокарты кластера в течении суток и была сохранена на следующих эпохах: 5 000, 40 000, 100 000, 200 000, 1 200 000.

Во время исследования были взяты три случайных вектора (распределение равномерное). После чего один и тот же вектор подаётся на вход всем моделям и оценивается результат.

\subsubsection{5 000 ЭПОХ}
\begin{enumerate}
    \item Очень много нот и повторений, нет вариативности.
    \item Имеет классическую структуру (условно припев-куплет).
    \item Ритм-партия обычная.
    \item Композиции 1, 2 и 3 друг друга повторяют, т.е практически ничем не отличаются.
    \item Присутствует огромное количество лишних звуков, которые обычно редко используются.
\end{enumerate}

Вывод: на данном этапе модель смоделировала базовую структуру композиции и ритм-секции (бас-гитара и барабаны).

\subsubsection{40 000 ЭПОХ}
\begin{enumerate}
    \item Начала появляться выраженная мелодия композиции.
    \item Ритм-секция стала разнообразной.
    \item Появилась малая вариативность партий каждого из инструментов; они друг друга немного дополняют.
\end{enumerate}

Вывод: на данном этапе модель сгенерировала вариативность в куплетах и паузы для некоторых инструментов.

\subsubsection{100 000 ЭПОХ}
\begin{enumerate}
    \item Появилось очень много акцентов в ритм-секции.
    \item Инструменты друг друга дополняют.
    \item Появилось небольшое разнообразие партий инструментов в течение всей композиции.
\end{enumerate}

Вывод: на данном этапе модель смоделировала полифонию между инструментами и акценты композиции.

\subsubsection{200 000 ЭПОХ}
\begin{enumerate}
    \item Акцентов стало заметно меньше, они стали менее выделяющимися.
    \item Появилась основная тема мелодии, которая присутствует в течение всей композиции.
    \item Модель чаще заходит в тупик (генерирует одно и то же для инструмента, долго не может выйти из этого состояния). Партия бас-гитары характеризуется дисгармоничным звучанием и однообразием. 
\end{enumerate}

Вывод: на данном этапе модель более точно сгенерировала  акценты композиций и наличие основной темы. 

\subsubsection{1 200 000 ЭПОХ}
\begin{enumerate}
    \item Присутствует некоторое развитие композиции. 
    \item Переходы между частями композиции стали не мгновенными.
    \item Модель меньше заходит в тупик.
    \item Композиции отличаются.
    \item Появились редкие явные акценты.
\end{enumerate}

Вывод: на данной этапе модель смоделировала редкие явные акценты и плавные переходы между частями композиций.

\subsection{ОПРЕДЕЛЕНИЕ ЖАНРА КОМПОЗИЦИЙ}
Для изучаемой модели данный вид определения наиболее правильной композиции с помощью MP3 файла не подходит, т. к. на всех представленных этапов обучения вероятность принадлежности жанру меняется крайне мало. Это можно увидеть в таблицах, которые представлены ниже в разделе MuseGAN на сайте, представленном в главе \ref{chapter31}. Вполне вероятно, что это связано с тем, что в результате модель генерирует композицию из одного и того же распределения (и при анализе модели это видно) или получается совсем другое звучание, которое не ожидает модель определения жанра.

В результате анализа на жанр MIDI-файла получается та же ситуация, что и с MP3, следовательно, проблема не заключается в том, как преобразовывается MIDI-файл в MP3.


\section{THEME TRANSFORMER}
В рамках данной работы была сделана копия (описан в главе \ref{chapter31}) исходного удалённого репозитория для добавления возможности тренировки модели на нескольких видеокартах на кластере  \cite{theme_source}. Для реализации была выбрана библиотека для глубокого обучения pytorch-lightning, т. к. она содержит большое количество утилит, в том числе для отслеживания процесса обучения, его конфигурации \cite{torch_lightning}.

Архитектурные отличия данной модели от MuseGAN были рассмотрены в теоретической части. Если в прошлом разделе был изучен процесс обучения модели для создания композиции с нуля, то сейчас рассмотрим генерацию с подсказкой -- композицией, которую необходимо продолжить. Вследствие этого стоит понимать, что результат будет разительно отличаться от предыдущей модели, т. к. процесс обучения отличается из-за особенностей архитектуры. Так же данная модель генерирует результат только для двух инструментов, что облегчает процесс создания полифонической музыки. 

В результате работы модели получаем композиции, состоящие из трёх дорожек -- основной мелодии, аккомпанемента и пометки, где встречается повтор композиции, которую подавали на вход. 


\subsection{ПРОЦЕСС ОБУЧЕНИЯ}
Данная модель обучалась в течении двух дней и была сохранена на следующих эпохах:
\begin{enumerate}
    \item 214 эпохи (25 минут)
    \item 300 эпох (35 минут)
    \item 514 эпох (1 час)
    \item 771 эпоха (1 час 30 минут)
    \item 15000 эпох (29 часов)
    \item 20290 эпох (48 часов)
\end{enumerate}

Рассмотрим процесс её обучения.

\subsubsection{25 МИНУТ}
\begin{enumerate}
    \item Модель с первых секунд нарушает тональность мелодии.
    \item С течением композиции модель начинает реже добавлять ноты.
    \item Основная тема композиции звучит иначе.
    \item Ближе к концу модель будто оканчивает композицию, однако, т. к. модель не может заканчивать композицию в связи с особенностями архитектуры, ей приходится генерировать продолжение.
\end{enumerate}

\subsubsection{35 МИНУТ}
\begin{enumerate}
    \item Значение лосс-функции на валидационной выборке минимальное.
    \item Модель ничего не получила на выходе из-за огромного количества ошибок генерации; в основном -- нарушение размера ритма. 
\end{enumerate}
    
\subsubsection{1 ЧАС}
\begin{enumerate}
    \item Основная тема звучит немного иначе.
    \item Мелодия разнообразна как ритмически, так и мелодически.
    \item Некоторые ноты нарушают тональность, что свидетельствует о том, что модель не выучила до конца правила языка.
\end{enumerate}
  
\subsubsection{1 ЧАС 30 МИНУТ}
\begin{enumerate}
    \item Основная тема всё ещё звучит немного иначе.
    \item Присутствуют интересные ритмические приёмы.
    \item Изредка нарушается тональность.
    \item Композиция не замедляется.
    \item Модель не стремится закончить композицию, как в предыдущих случаях.
\end{enumerate}
   
\subsubsection{29 ЧАСОВ}
\begin{enumerate}
    \item Основная тема звучит нормально и модель чаще использует её в композиции.
    \item Нет нарушений тональности, следовательно модель изучила правила языка. 
\end{enumerate}

\subsubsection{48 ЧАСОВ}
\begin{enumerate}
    \item Основная тема звучит немного по-другому.
    \item Модель намного чаще вставляет ноты, чаще появляются интересные мелодии, так же вставляется основная тема в композицию.
\end{enumerate}

\subsection{ОПРЕДЕЛЕНИЕ ЖАНРА КОМПОЗИЦИЙ}
При определении жанра с помощью MP3 у данной модели схожая ситуация с MuseGAN. Жанр определяется как джаз вне зависимости от качества трека (качество в данном случае -- субъективная оценка).

При определении жанра с помощью MIDI всегда получается один и тот же вектор вероятностей.

\section{СОЕДИНЕНИЕ МОДЕЛЕЙ MUSEGAN И THEME TRANSFORMER}
Т. к. одна модель генерирует композицию с нуля, а другая продолжает, то их можно соединить. 
Для их соединения на результат модели MuseGAN необходимо выполнить следующую предобработку, чтобы подать Theme Transformer на вход.
\begin{enumerate}
    \item Обрезаем MIDI-файл до 6 первых секунд.
    \item Оставляем только две дорожки: Guitar и Piano.
    \item Переименовываем их в MELODY и PIANO соответственно.
    \item Вставляем ещё одну дорожку, в которой будет информация об основной теме композиции длиною в весь получившийся файл.
\end{enumerate}

В итоге получается правильный вход для Theme Transformer из выхода MuseGAN и благодаря такой обработке можно соединить две модели.
Результат полученной модели так же представлен на сайте (описан в главе \ref{chapter31}) в разделе MuseGAN + Theme Transformer. В заголовках указана, какая модель Theme Transformer используется, в первом столбце -- вид модели MuseGAN. 1, 2 и 3 -- соответствующие случайные вектора, подающиеся на вход моделям MuseGAN.

В случае, если композиции в ячейке нет или она меньше 30 секунд, модель Theme Transformer заходит в тупик и генерирует неверную по структуре композицию.

\subsubsection{АНАЛИЗ РЕЗУЛЬТАТОВ ГЕНЕРИРОВАНИЯ}
Описанные модели достаточно хорошо справляются с задачей генерирования композиций. Действительно, их архитектура позволяет моделям создавать временную структуру и учить языковую модель.

Проведя исследование, можно сделать вывод, что Theme Transformer сильно зависит от результатов работы первой модели. MuseGAN часто в начале генерирует большое количество нот. Вследствие этого модель может повторять многократно данную тему в точности, что плохо сказывается на результате. Заместо данного эффекта необходимо повторение исходной темы в разных вариациях, благодаря чему получается разнообразный контент. Данный эффект достигается при применении модели, которая обучалась 48 часов. 

Чем больше обучалась модель, тем больше нот генерирует Theme Transformer и композиция становится более стремительной (крайне мало расстояние между нотами в мелодии, ноты часто меняются, мелодия быстро меняется). И наоборот, чем меньше обучалась модель, тем больше пауз, замедления в генерируемом продолжении.

Результат практически не зависит от типа модели MuseGAN. 

С помощью соединения моделей получилось генерировать более разнообразный контент. Так же классификатор для определения жанра с помощью MP3 начал обнаруживать небольшие различия в генерируемом контенте, что может означать разнообразие результатов.

\section{ВАРИАНТЫ УЛУЧШЕНИЯ МОДЕЛЕЙ}

Для обеих моделей возможны некоторые модификации, которые, на первый взгляд, являются хорошим дополнением. Среди таких модификаций можно назвать вариационный кодировщик и использование дополнительной модели с определением жанра.

Вариационный кодировщик позволяет строить более гладкие поверхности без сильных изменений, что может отобразиться на качестве генерируемого материала. Однако данную модификацию, скорее всего, удастся применить только в архитектуре трансформеров.

Использование дополнительной модели для выбора поможет среди всего разнообразия генерируемого материала найти те, которые действительно можно отнести к какому-либо жанру. Если применять данную идею без дообучения классификатора, то реализованная идея работать не будет. Модели сами по себе генерируют с точки зрения классификаторов одно и то же.

Данную идею можно развить следующими способами.
\begin{enumerate}
    \item Обучить простую модель для бинарной классификации, которая сможет фильтровать результаты нейронной сети. Данный пункт актуален, т. к. определение жанров не является решением проблемы выбора.
    \item Соединить два вычислительных графа в один и дообучить модель. Перед реализацией необходимо подумать о том, как градиенты будут проходить по графу и имеет ли это смысл.
    \item Попробовать модифицировать архитектуру с помощью вариационного кодировщика.
\end{enumerate}
