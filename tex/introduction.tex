Нейронные сети на данный момент применяются во многих областях. Их применяют для распознавания образов, синтеза речи и её распознавания и т.д. Данная область растёт очень быстро и решает большой круг задач, которые раньше могли решить только люди. 

Так же нейронные сети могут применяться и в задаче создания музыки. Для этого необходимо знать некоторые правила создания композиций. Здесь можно привести пример с говорением. В устной речи тоже необходимо знать правила языка для того, чтобы грамотно излагать мысли. И в той, и в другой ситуации при нарушении правил могут возникнуть проблемы с восприятием. Если человек скажет что-то неграмотно, то другой может его не понять. Если будет сыграна нота, не относящаяся к тональности, то это будет 'резать' слух. 

Музыка -- это искусство, обладающее периодичностью, а потому требующее временной модели. Музыка имеет иерархическую структуру -- высокоуровневые строительные блоки (фразы) состоят из более мелких рекуррентных шаблонов (тактов). Когда люди слушают музыку, то они уделяют внимание структурным шаблонам, связанным с согласованностью, ритмом, напряжением и потоками эмоций. Таким образом, механизм учета временной структуры имеет решающее значение.

Музыка обычно состоит из нескольких инструментов/дорожек. Современный оркестр, как правило, включает в себя четыре различные инструментальные секции: духовые, струнные, деревянные духовые и перкуссия; рок группа часто состоит из баса, ударных, гитары и вокала. Эти дорожки тесно связаны и разворачиваются во времени независимо.

Ноты часто образуют аккорды, арпеджио или мелодии в полифонической музыке. В связи с этим введение хронологического порядка нот не совсем естественно, т. к. в композиции партия одного инструмента может дополнять партию другого. Таким образом, успехи в создании текстов на естественном языке и монофонической музыки не могут в полной мере распространяться на генерацию полифонической музыки.

Генерация музыкального контента может быть необходима для: музыкальных стриминговых сервисов под настроение и пожелания пользователя, музыкального сопровождения подкастов, видеороликов, игр и пр., создания идеи композиции, продолжения существующей композиции и т.д.

Генерация музыки уже успешно применяется в сервисах для предоставления композиций под пожелания пользователя \cite{brainfm}.

В данной работе рассмотрены алгоритмы генерации музыки с помощью генеративно-состязательных нейронных сетей и трансформеров. Так же были изучены данные модели на разных этапах тренировки (глава \ref{chapter4}) и было проверено эвристическое правило отбора 'лучших' композиций с помощью классификаторов жанров (глава \ref{chapter3}). 

В рамках анализа архитектуры (главы \ref{chapter1} и \ref{chapter2}) исследована каждая из моделей (MuseGAN, Theme Transformer соответственно) и выявлены их недостатки.
