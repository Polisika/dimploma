\chapter*{АННОТАЦИЯ}
Отчет 54 с., 5 ч., 8 рис., 18 источников.

ГЛУБОКОЕ ОБУЧЕНИЕ, МУЗЫКА, MIDI-ФАЙЛЫ, НЕЙРОННЫЕ СЕТИ, АРХИТЕКТУРА ТРАНСФОРМЕРОВ, ГЕНЕРАТИВНО-СОСТЯЗАТЕЛЬНЫЕ НЕЙРОННЫЕ СЕТИ, ГЕНЕРАЦИЯ МУЗЫКИ

Объектом исследования являются алгоритмы генерации музыки, использующие нейронные сети.

Цель работы – проведение сравнительного анализа алгоритмов генерации музыки с использованием нейронных сетей для выявления процесса обучения и разнообразности результата.

В процессе работы проводилось изучение архитектур нейронных сетей MuseGAN и Theme Transformer, которые успешно генерировали композиции. Так же была рассмотрена такая проблема, как разнообразие генерируемых композиций с помощью классификаторов жанров. 

В результате исследования алгоритмов для генерации музыки были исследованы такие модели, как MuseGAN и Theme Transformer. В результате исследований было выявлено, что данные модели генерируют с точки зрения классификаторов жанров одно и то же (первый и второй классификаторы используют MP3 и MIDI файлы для анализа соответственно).

Также был предложен новый вариант алгоритма, который подразумевает под собой комбинацию данных моделей — MuseGAN генерирует начало композиции, а Theme Transformer её продолжает. Такое решение генерирует более разнообразные композиции (с точки зрения первого классификатора). Был создан сайт для прослушивания результатов генерации и разработан программный модуль на языке Python для исследований. 
